\documentclass[11pt,
  titlepage=false,
  abstract=on,
  %parskip=half,      % enable if you want paragraphs separated by vertical spacing instead of indents
]{scrreprt}

% Style settings
\usepackage[utf8]{inputenc}
\usepackage{microtype}
\usepackage[acronym]{glossaries}
\makeglossaries
\addtokomafont{disposition}{\rmfamily}
\renewcommand*\thesection{\arabic{section}}

% Useful packages for complex content:
\usepackage{amsmath,amsfonts,amssymb} % typesetting math
%\usepackage{siunitx}                 % typesetting SI-units and formatted numbers
%\usepackage{listings}                % typesetting source code
\usepackage{booktabs,multirow}        % utils for complex/beautiful tables
%\usepackage{subcaption}              % placing multiple subfigures in a figure
%\usepackage{graphicx}                % including external image files
%\usepackage{tikz}                    % drawing figures within LaTeX

% Bibliography, referencing, and indexing
\usepackage{csquotes}                 % typesetting \enquote{text in quotes} correctly
\usepackage[backend=biber,
            style=alphabetic,
            minalphanames=3, maxalphanames=4,
            maxbibnames=20]{biblatex} % to generate the bibliography
\addbibresource{report.bib}           % name of the bib-file

% Useful utils:
%\usepackage{todonotes}               % add ToDo markers (\todo{...}, \todo[inline]{...})
\usepackage[hidelinks]{hyperref}      % clickable links (but hide color frames around links)
%\usepackage{cleveref}                % named references (\Cref{sec:introduction}, ...)

% Your own macros:
%\newcommand{\mynewmacro}[1]{my content with one input parameter: #1}


\begin{document}

%--- FRONT MATTER --------------------------------------------------------------

\title{Evaluation of modular multiplication methods in lattice-based cryptography on FPGA/ASIC}
\author{Teodora-Alexandra Alexandrescu}
\date{Introduction to Scientific Working 2024/25\bigskip\bigskip}
\publishers{\normalsize
  Supervisor:
  Aikata
  \medskip\par
  Institute of Applied Information Processing and Communications\\
  Graz University of Technology
}

\maketitle


\begin{abstract}\noindent
Lattice-based cryptography is the backbone for an entire class of systems offering post-quantum security.\\
Many of these cryptosystems require rigorous mathematical approaches for laying the foundation of security, especially in the case of 
post-quantum computing or fully-homomorphic encryption schemes. A well-known NP-hard problem that is fundamental for both previously mentioned algorithms
is Ring Learning with Errors, which uses polynomial arithmetic. In several cases, if these polynomials are small and a naive computational method is used,
the performance costs will also be minimal. However, the higher the polynomial degree, the higher the need for optimization becomes.\\
To solve this problem, the concept of modular multiplication has been applied, where each coefficient in the polynomial result needs to be reduced modulo a quotient $q$ in order to reduce the coefficients. Thus, modular multiplication is the most fundamental resource in cryptographic schemes. This arithmetic operation, 
if performed efficiently, can bring a significant impact over the performance of lattice-based cryptographic implementations.\\
In this work, we address the problem of efficiency in terms of area and power consumption on digital platforms such as FPGA/ASIC
and analyze three well-known modular multipliers for evaluation: Montgomery Reduction, Barret's Reduction and Add \& Shift. Furthermore, we seek to distinguish between two novel usecases of lattice-based cryptography,
namely Post-Quantum Cryptography and Fully-Homomorphic Encryption, based on how they utilize the NP-Hard Ring Learning with Errors problem.

% Provide an abstract of your report (at most $\frac{1}{2}$ page for this report, typically 1 to 3 paragraphs).

% The abstract usually consists of two main parts: a motivational background and your contribution.
% Start with a few sentences of general introduction and background information to motivate your main research question/challenge.
% Then, summarize what your paper contributes and describe its (potential) impact.
% This includes a very short summary of all your important results and core performance numbers that characterize your approach/attack/countermeasure/implementation.
% Finally, summarize any key conclusions and calls to action that you have, e.g., apply the idea more broadly, get rid of some technology, find a countermeasure, or similar.


\paragraph*{Keywords:}
Lattice-based Cryptography $\cdot$
Post-Quantum Cryptography$\cdot$
Fully-Homomorphic Encryption$\cdot$
Modular Multipliers $\cdot$
FPGA $\cdot$
ASIC
\end{abstract}

\clearpage


%--- INTRODUCTION --------------------------------------------------------------

\section{Introduction}
\label{sec:introduction}
% To further explore the intricacies of these cryptographic schemes, 
% In asymmetric cryptography, ensuring that the private key remains secret under any circumstances is crucial. (\texttt{not entirely sure about this, 
% there's no punchline and it doesn't catch the attention of the reader. for sure an actual scientific question must be stated.})

Lattice-based constructions have gained notoriety in recent times due to their complexity, a characteristic that correlates with enhanced security.
As the need for secure and lightweight lattice-based algorithms increases, their complexity is also an aspect that requires intensive study, especially
in the context of platforms like FPGA or ASIC, where area and power are limited resources.
At the heart of many lattice-based constructions lies polynomial multiplication, which is the basic and most computationally intensive operation in 
Ring Learning with Errors cryptosystems. One crucial aspect of polynomial multiplication in lattice-based cryptography is modular multiplication, which 
ensures that the polynomial coefficients lie within manageable bounds while still keeping the initial security properties. However, the higher the security 
degree, the more convoluted these operations become and hence resources such as occupied area or power consumption need to be preserved.

For this reason, we strive to provide an evaluation of modular multiplication methods targeted on digital platforms, in order to be aware of possible bottlenecks 
or caveats during the implementation of such schemes. To further mitigate these potential losses, we need to firstly understand the complexity of the operations
performed within these methods.

% \{ \texttt{I think that 90 \% of the follwing paragraphs need to be moved into the background section because there's too many details}\}
Modern-day experts base their encryption algorithms on hard mathematical problems such as the Learning with Errors problem to guarantee security within a certain standard.\\
After 2005, the Learning with Errors problem published by Oded Regev \cite{regev2010learning} became a standard for cryptographic constructions due the strict security requirements it enforces.
A study made by Micciancio \& Peikert proves its hardness to resemble the norms of approximate lattice problems in the worst case \cite{micciancio2013hardness}.
% It assumes a set of random linear equations over a secret vector $\textbf{s}$. Each equation is of form $\textbf{a} \cdot \textbf{s} + e$, where $\textbf{a}$ is the public
% vector, $\textbf{s}$ is the secret vector and $e$ is the error term drawn from a Gaussian distribution.
% While it may be possible to have access to $\textbf{s}$ with the multiplication between $\textbf{a}$ and $\textbf{s}$ alone, the error term makes it computationally 
% infeasible to recover the secret $\textbf{s}$ for appropriate parameters.\\
% \texttt{I think that the following block of text is too much into detail and could be``abstracted''}
The problem states that a secret vector $\textbf{s}$ should be retrieved from a system of linear noisy equations, where the public vector $\textbf{a}$ is drawn uniformly at random
and the error term $\textit{e}$ is drawn from a Gaussian distribution. While it may be possible to recover the secret $\textbf{s}$
by performing Gaussian decomposition over the system of equations, the addition of the noise term makes it computationally infeasible to retrieve the
secret for appropriate parameters.

% \{The original problem states that a secret vector $\textbf{s}$ from a \textit{'noisy'} system of linear equations of type $\textbf{a} \cdot \textbf{s} + e$ should be retrieved, where $\textbf{a}$ is the public
% vector drawn uniformly at random and $\textbf{e}$ is the error term or the $noise$, drawn from a Gaussian distribution. While it may be possible to recover the secret $\textbf{s}$
% by performing Gaussian decomposition over the system of equations, the addition of the noise term makes it computationally infeasible to retrieve the
% secret for appropriate parameters.\} Thus, the main challenge that the attacker faces is distinguishing ``random noisy equations'' from truly uniform ones.

% from $\textbf{a} \cdot \textbf{s}$ alone 

The parameters provided to Learning with Errors, either number of vectors or vector dimension, usually determine the complexity of the encryption operation \cite{micciancio2013hardness}, or more specifically, the key size. 
Cryptographic applications often need at least  $n$ vectors of $n$ dimensions, which lead to key sizes of $n^2$ \cite{regev2010learning}. This aspect can represent a challenge as the need for a higher 
security level increases, which makes the problem inefficient for practical cryptographic applications.
% problem
% Nevertheless, 
% cryptographic schemes based on \gls{LWE} tend to require large key sizes. 
% this highly depends on the desired security level \textit{n} which also determines the ``hardness'' . 
% Cryptographic applications often need at least  $n$ vectors $\textbf{a}_1,\ldots,\textbf{a}_n$, which leads to key sizes of $n^2$ \cite{regev2010learning}. This aspect can represent a challenge as the need for a higher 
% security level increases, which makes the problem inefficient for practical cryptographic applications. A solution to this issue is tightly related to the NTRU problem, which is based on the mathematical structure of
% polynomials over rings. 
The Ring Learning with Errors problem \cite{lyubashevsky2010ideal} was introduced by Lyubashevsky as an optimization of the initial LWE problem.

Lyubashevsky's solution suggests a more efficient and secure approach in implementing encryption algorithms by applying the heuristic design of the NTRU cryptosystem, the ring structure over a finite field of moduli, over the polynomial representation 
of public and private keys. With that being said, using polynomial multiplication within a ring structure instead of vectors and inner products should be more lightweight from a computational point of view.
However, if we were to plainly multiply each polynomial with another, we would end up doing modulo multiplication of all coefficients with each other. When combining this method with the large parameters passed to the RLWE
problem, the computations become highly convoluted. For this reason, more advanced modular multiplication methods have been introduced, such as Montgomery Reduction or Barrett Reduction, which we are discussing in the context of Post-Quantum Cryptography and 
Fully-Homomorphic Encryption. Moreover, we are also discussing the Add \& Shift method in combination with modular reduction, which is the basis for other modular reduction methods.

% \texttt{(the following sentences are just placeholders for now)}


% \paragraph{Our contributions: }In this paper, we want to explore these algorithms, while aiming to create an evaluation of each and every one of them in different scenarios within lattice-based cryptographic methods.
% , which facilitates
% the usage of RLWE in complex lattice-based applications such as Quantum-Safe Cryptographic Schemes and Fully-Homomorphic Encryption.

% Lattice-based applications such as Quantum-Safe Schemes and Fully-Homomorphic Encryption make use of the polynomial representation within the ring for expressing asymmetric designs. H
% However, polynomial multiplication is still an expensive operation, especially in the context of devices with limited resources such as FPGA or ASIC. The basic encryption operations require a set of RLWE parameters which define the computational properties
% of the ring structure and determine the level of security (i.e. polynomial degree). Even though the security level can be optimized, the lowest acceptable security level still yields a large ring object, of the order of millions of bits. (add reference here)
% The ring structure not only reinforces the "hardness" of 
% the initial problem, but also allows for operations on encrypted data to be performed. Both of these features facilitate the implementation of post-quantum cryptographic
% algorithms and fully-homomorphic encryption schemes.
%  by "exploiting" the structure of the LWE samples, which results in a ring over a finite
% field of moduli. The ring structure reduces the information needed to describe each vector $\textbf{a}_i$ by deriving each one of them from $\textbf{a}_1$ - this reduces the key sizes from $n^2$ to $n$. 

% ($F_q[X] / \Phi[X]$ \texttt{maybe this belongs better in the background?})

% Provide an introduction to your work (typically 1 to 2 pages, but can also be more for your full thesis).
% For this short ISW report, focus on \textbf{Introduction + Background + Conclusion} and cover them it at least \textbf{4 pages} (excluding the titlepage/abstract and bibliography).

% The introduction is structured like a longer version of the abstract.
% You start with a paragraph or two of general background, getting more and more specific, and culminating in a clear motivation for your main research problem or thesis goal.
% This should be a convincing and well-founded story: Readers will stop reading here if they do not see the relevance of your work for them.
% If you have a central research question, state it clearly.
% If there are particular papers relevant to your work, such as central techniques you used, designs you built on or evaluated, or similar, be sure to include citations here.

% Then, provide a summary and relevant details to characterize your contribution and approach for solving the research problem.
% The introduction is generally written in present tense and active form -- everything happens now and is done by ``we'', the author(s):
% ``We provide an implementation'', ``We discuss'', ``We evaluate'', etc.
% The focus is on your final \emph{outcome} and gained insights, not on your personal journey to get there; for example, do include facts like the runtime of your implementation if it is relevant for judging its performance, but do not include how many months it took you to produce this implementation.
% Provide a concise summary of your approach and your main findings, including all details already given in the Abstract and some more details.
% This includes a clear definition of the scope of your work and your assumptions, such as your attacker model, hardware/software used, and similar.
% There should be no big (positive or negative) surprises on your results for the reader in the remaining paper, except for technical details on \emph{how} you achieved your results.
% If there is a useful central figure that helps explain and contextualize your contribution, or a summary table that compares your contribution with related work, you can add it here.
% Finish this part with a (bulleted) list of around 3 to 5 main scientific contributions, such as new ideas and techniques, applications of techniques to new application areas, novel implementation results, performance numbers, and similar.

% At some point in your thesis, you need to discuss relevant related work.
% Depending on what your work contributes, a good spot for this discussion can be in the \emph{Introduction}, in the \emph{Background} section, or in a final \emph{Discussion} section.

\paragraph{Outline.}
% The introduction ends with an outline (aka mapping), explaining how this paper is structured:
% In Section \ref{sec:background}, we introduce the relevant background on some topic.
% (\dots)
% Finally, in Section \ref{sec:conclusion}, we conclude with a discussion of our findings and directions for future work.


%--- BACKGROUND ----------------------------------------------------------------

\section{Background}
\label{sec:background}

\subsection{Notation}
We use $\mathbb{Z}_q$ to represent the set of integers modulo a quotient $q$. We represent the set of $n$-dimensional vectors with $\mathbb{Z}^n_q$
\subsection{Briefly on to Lattices}
\subsection{Learning with Errors}
\paragraph{Output}
We denote the Learning with Errors problem as following: $(\mathbf{a}, \langle \mathbf{a}, \mathbf{s} \rangle + e)$, where $\mathbf{a}$ is a vector
drawn uniformly at random from $\mathbb{Z}^n_q$. The goal is to retrieve the secret vector $\mathbf{s} \in \mathbb{Z}^n_q$ given an error term
$e$ drawn from an error.

\subsection{NTRU}
\subsection{Ring Learning with Errors}
\subsection{Polynomial multiplication}
\subsection{Modular multiplication}
\subsection{Montgomery Reduction}
\subsection{Barrett Reduction}
\subsection{Add \& Shift}
% Ring Learning with Errors is at the core of many lattice-based algorithms and involves the manipulation of ring objects in polynomial form.

% Every section starts with an introductory paragraph that outlines the contents and purpose of the section:

% In this section, we provide some usage examples for bibliography and citations with \texttt{biblatex} (Section \ref{subsec:bib}) and writing tips (Section \ref{subsec:hints}).

% \subsection{Citations}
% \label{subsec:bib}

% This is an example of how to specify and cite
% a book \cite{AESbook},
% a journal article \cite{bstjShannon49},
% a conference article \cite{spKocherHFGGHHLM019},
% an informal report \cite{iacrSchneierFKR15},
% and a website \cite{webIAIK21}.
% We can also add the authors' names to the citation:
% AES is a block cipher defined by \textcite{AESbook}.

% \subsection{Writing Style}
% \label{subsec:hints}

% Writing style recommendations for English differ a bit from German:
% \begin{itemize}
%   \item Prefer short, clear sentences over long, convoluted sentences.
%   \item Prefer active voice, well-defined subjects, and meaningful verbs over passive voice, vague subjects, and vague verbs.
%   \item Do not concatenate too many nouns.
%   \item When you refer to the same thing several times consecutively, call it by the same name instead of inventing synonyms. (``Wortwiederholung'' is ok.)
% \end{itemize}
% Consider using a style checker such as Grammarly -- ask your supervisor for details.

% Rules for placing commas in English are quite easy, but also quite different from German.
% The most important ones include:
% \begin{itemize}
%   \item No comma before ``that'': ``Due to the fact that\dots''
%   \item No comma before infinitives: ``We did this to find out\dots''
%   \item No comma between subject and verb (be careful not to confuse with introductory clauses): ``Completing the list is essential.''
%   \item No comma before indispensable relative clauses: ``The function returns the key which has the highest score.''
%   \item Put a comma before and after dispensable relative clauses and other non-essential phrases: ``The key, which consists of two subkeys, is generated\dots''
%   \item Put a comma after introductory clauses: ``Consequently, we use\dots''; ``After the last step, we return\dots''; ``To complete the list, we add\dots''.
%   \item Put a comma before and after ``e.g.'' and ``i.e.'': ``A block cipher, e.g., AES.''
%     Consider using ``for example/for instance/such as'' and ``that is'' instead.
%   \item Many authors put a comma before ``and, but, for, or, nor, so, yet'' when they connect two independent clauses: ``We repeated the experiment, but the result was different.''
%     This is a matter of (your or your supervisor's) preference.
%   \item Many authors put a comma before ``and, or'' in the last element of an enumeration of three or more elements: ``Alice and Bob'', but ``Alice, Bob, and Caesar.''
%     This ``Oxford comma'' is a matter of (your or your supervisor's) preference.
% \end{itemize}

% Typesetting and other hints:
% \begin{itemize}
%   \item To typeset English `single' and ``double'' quotation marks in \LaTeX, start with \verb|`| (grave accent) and end with \verb|'| (typewriter apostrophe).
%   \item Display math formulas are introduced in the previous sentence and include punctuation (often separated by a thin space \verb|\,|): The ciphertext is computed as
%     \[C = E_K(P)\,.\]
%   \item Captions are usually above tables, but below figures, and end with a full stop.
%   \item ``This key has 128 bits'', but ``128-bit key''.
%   \item References are typically capitalized and separated by a non-breaking space \verb|~|: ``Section~\ref{subsec:bib}''.
%     Try \verb|\autoref| (\autoref{subsec:bib}) or the \verb|cleveref| package. % \Cref{sec:bib}
% \end{itemize}

% Figure \ref{fig:diagram} illustrates the data of Table \ref{tab:data}.

% \begin{figure}[htpb]
%   \centering
%   %\includegraphics[width=.5\textwidth]{figures/myimage.png}
%   \caption{Concise descriptive caption for the figure (printed below the figure).}
%   \label{fig:diagram}
% \end{figure}

% \begin{table}[b]
%   \caption{Concise descriptive caption for the table (printed above the table).}
%   \label{tab:data}
%   \centering
%   \begin{tabular}{lrr} % text columns: {l}eft-aligned, number columns: {r}ight-aligned
%     \toprule
%     Item   & \multicolumn{2}{c}{Properties} \\
%              \cmidrule{2-3}
%            & First          & Second \\
%     \midrule
%     Apple  & 5              & 100 \\
%     Pear   & 10             & 99 \\
%     \bottomrule
%   \end{tabular}
% \end{table}


%--- CONCLUSION ----------------------------------------------------------------

\section{Conclusion}
\label{sec:conclusion}

% Provide the conclusions of your short report (max. $\frac{1}{2}$ page).

% This is structured similarly to the abstract and introduction.
% However, unlike the abstract, it can be partially written in past tense (for actions you performed and results that you found) as well as future tense or conditional (for predictions what the impact of your work will be).
% Start by briefly recalling the main motivation and main goal of your work.
% Repeat the main hard facts, performance numbers, and properties of your solution/work.
% %
% Emphasize your main insights, findings, and lessons learned.
% If you do not have a dedicated discussion section, you can discuss your findings and put them into perspective.
% If you have any recommendations based on your work, phrase them here.

% Finally, you can point out open problems that call for future work, but phrased in a positive way -- as opportunities.
% There should be no complete surprises (such as significant shortcomings not discussed earlier), but you can provide some new thoughts and ideas.
% %--- ACRONYMS --------------------------------------------------------------
% \newacronym{RLWE}{RLWE}{Ring Learning with Errors}
% \newacronym{LWE}{LWE}{Learning with Errors}
% \glsaddall
% \printglossary[type=\acronymtype, title={List of abbreviations}, nonumberlist]
%--- BIBLIOGRAPHY --------------------------------------------------------------

\printbibliography[heading=subbibliography]



\end{document}
